\documentclass[12pt, oneside, openany]{article}
\usepackage[T1]{fontenc}
\usepackage[spanish, es-tabla, es-lcroman]{babel}
\usepackage[utf8]{inputenc}
\usepackage[document]{ragged2e}
\usepackage{tcolorbox}
\tcbuselibrary{theorems}
\usepackage{cancel}
\usepackage{amssymb}
\usepackage{amsmath}
\usepackage{mathrsfs}
\usepackage{wrapfig}
\usepackage{fancyhdr}
\usepackage{colortbl}
\usepackage{graphicx}
\usepackage{subcaption}
\usepackage{xcolor}
\usepackage{tikz}
\usetikzlibrary{positioning}
\usepackage{multicol}
\usepackage{multirow}
\usepackage{lastpage}
\usepackage{pdfpages}
\usepackage{listings}
\usepackage{blindtext}
\spanishdecimal{.}
\usepackage[explicit]{titlesec}
\usepackage[colorlinks=true, linkcolor=black, citecolor=black, urlcolor=blue]{hyperref}
\usepackage[a4paper, total={16cm, 24cm}]{geometry}
\pagestyle{fancy}
\lhead{Muñoz Nuñez Ian Emmanuel}
\rhead{Proyecto 4}
\lfoot{Mtra. María Patricia Ventura Nuñez}
\rfoot{CUCEI}
\renewcommand{\headrulewidth}{1pt}
\renewcommand{\footrulewidth}{1pt}

\setlength{\headheight}{14.49998pt}

\begin{document}

\begin{titlepage}
    \pagenumbering{roman}
    \centering
    {\bfseries\LARGE Universidad de Guadalajara \par}
    \vfill
    {
        \includegraphics[width=0.3\linewidth]{UdG.png}
        \includegraphics[width=0.3\linewidth]{qci.png}
        \par
    }
    \vfill
    {\bfseries\LARGE Seminario de problemas de programación de sistemas reconfigurables \par}
    \vfill
    {\bfseries\LARGE Proyecto 4 \par}
    \vfill
    {\LARGE Diseñar un circuito combinacional en donde aparezca en un display la palabra Seminario y su nombre o apellido utilizando la \emph{GAL22v10} \par}
    \vfill
    {\bfseries\LARGE Nombre: \par}
    \vfill
    {\bfseries\LARGE Muñoz Nuñez Ian Emmanuel \par}
    \vfill
    {\bfseries\LARGE Sección: D01 \par}
    \vfill
    {\bfseries\LARGE Código: 216464457 \par}
    \vfill
    {\bfseries\LARGE Maestra: \par}
    \vfill
    {\bfseries\LARGE María Patricia Ventura Nuñez \par}
    \vfill
    {\bfseries\LARGE Ingeniería robótica \par}
\end{titlepage}

\pagenumbering{arabic}
\newpage
\section{Objetivo}
{\sffamily\large
    \hspace{0.5cm} Solucionar problemas de diseño utilizando las herramientas aprendidas en programación de sistemas reconfigurables.
    
    \hspace{0.5cm} Utilizar hojas de datos de las familias lógicas.
    
    \hspace{0.5cm} Simular circuitos digitales en programas de diseño como \emph{Proteus\texttrademark} e implementarlos físicamente.
    
    \hspace{0.5cm} Diseño e implementación de una función con salidas múltiples utilizando el software Boole de Usto.
    
    \hspace{0.5cm} Ejemplo:
    \renewcommand{\labelitemi}{$\bullet$}
    \begin{itemize}
        \item Diseño de un decodificador BCD a nombre o código hexadecimal con salida en display utilizando una \emph{GAL22v10}.
    \end{itemize}
    
}

\section{Material}
{\sffamily\large
    \begin{itemize}
        \item Protoboard.
        \item Fuente VCC (5V).
        \item Resistencias de $200\Omega$ y $2k\Omega$.
        \item Dip switch de 8 bits.
        \item Display de 7 segmentos.
        \item GAL22v10.
    \end{itemize}
}

\newpage
\section{Marco teórico}
{\sffamily\large
    \subsection{Tabla de verdad}
    
    \hspace{0.5cm} Cada número binario se emparejo con cada uno de los caracteres que se querían, y luego se obtuvieron las ecuaciones para cada uno de los casos.
    
    \begin{table}[h!]
        \centering
        \sffamily
        \scalebox{1.8}{
        \begin{tabular}{|c||c|c|c|c||c|c|c|c|c|c|c|}
            \hline
              & w & x & y & z &  a & b & c & d & e & f & g \\
            \hline
            S & 0 & 0 & 0 & 0 &  1 & 0 & 1 & 1 & 0 & 1 & 1 \\
            \hline
            E & 0 & 0 & 0 & 1 &  1 & 0 & 0 & 1 & 1 & 1 & 1 \\
            \hline
            3 & 0 & 0 & 1 & 0 &  1 & 1 & 1 & 1 & 0 & 0 & 1 \\
            \hline
            I & 0 & 0 & 1 & 1 &  0 & 1 & 1 & 0 & 0 & 0 & 0 \\
            \hline
            n & 0 & 1 & 0 & 0 &  0 & 0 & 1 & 0 & 1 & 0 & 1 \\
            \hline
            A & 0 & 1 & 0 & 1 &  1 & 1 & 1 & 0 & 1 & 1 & 1 \\
            \hline
            r & 0 & 1 & 1 & 0 &  0 & 0 & 0 & 0 & 1 & 0 & 1 \\
            \hline
            I & 0 & 1 & 1 & 1 &  0 & 1 & 1 & 0 & 0 & 0 & 0 \\
            \hline
            O & 1 & 0 & 0 & 0 &  1 & 1 & 1 & 1 & 1 & 1 & 0 \\
            \hline
            n & 1 & 0 & 0 & 1 &  0 & 0 & 1 & 0 & 1 & 0 & 1 \\
            \hline
            U & 1 & 0 & 1 & 0 &  0 & 1 & 1 & 1 & 1 & 1 & 0 \\
            \hline
            ñ & 1 & 0 & 1 & 1 &  1 & 0 & 1 & 0 & 1 & 0 & 1 \\
            \hline
            E & 1 & 1 & 0 & 0 &  1 & 0 & 0 & 1 & 1 & 1 & 1 \\
            \hline
            Z & 1 & 1 & 0 & 1 &  1 & 1 & 0 & 1 & 1 & 0 & 1 \\
            \hline
              & 1 & 1 & 1 & 0 &  x & x & x & x & x & x & x \\
            \hline
              & 1 & 1 & 1 & 1 &  x & x & x & x & x & x & x \\
            \hline
        \end{tabular}
        }
        \caption{\sffamily Tabla de verdad del circuito}
        \label{tab:my_label}
    \end{table}
    
}

\newpage
\subsection{Ecuaciones lógicas}

\begin{equation*}
    a = (\overline{W}\,\overline{X}\,\overline{Z})+(WYZ)+(X\overline{Y}\,Z)+(\overline{W}\,\overline{Y}\,Z)+(W\overline{Y}\,\overline{Z})
\end{equation*}

\begin{equation*}
    b = (XZ)+(W\overline{X}\,\overline{Z})+(\overline{W}\,\overline{X}\,Y)
\end{equation*}

\begin{equation*}
    c = (W\overline{X})+(YZ)+(\overline{X}\,\overline{Z})+(\overline{W}\,X\overline{Y})
\end{equation*}

\begin{equation*}
    d = (\overline{X}\,\overline{Z})+(\overline{W}\,\overline{X}\,\overline{Y})+(WX)
\end{equation*}

\begin{equation*}
    e = (X\overline{Z})+(\overline{Y}\,Z)+W
\end{equation*}

\begin{equation*}
    f = (\overline{W}\,\overline{Y}\,Z)+(W\overline{Z})+(\overline{W}\,\overline{X}\,\overline{Y})
\end{equation*}

\begin{equation*}
    g = (\overline{W}\,\overline{Z})+(WZ)+(\overline{W}\,\overline{Y})+(X\overline{Y})
\end{equation*}

\section{Procedimiento}
{\sffamily\large
    \hspace{0.5cm} Para realizar el proyecto se conecto el dip switch a las primeras 4 entradas de la \emph{GAL22v10} y en paralelo a \emph{tierra} con resistencias de $2k\Omega$. Las primeras 8 salidas de la \emph{GAL22v10} se conectaron con resistencias de $200\Omega$ al display.
    
    \hspace{0.5cm} Los materiales utilizados son: 1 dip switch de 8 bits, 4 resistencias de $2k\Omega$ y 7 resistencias de $220\Omega$, por último, un display de 7 segmentos y una \emph{GAL22v10}.
    
}

\section{Circuito a implementar}

\subsection{Simulación}
{\sffamily\large
    \hspace{0.5cm} En la siguiente página se muestra la simulación del circuito a implementar.
    
    \newpage
    \includepdf[pages={1}]{main.PDF}
    
}

\subsection{Protoboard}
\begin{figure}[h!]
    \centering
    \begin{subfigure}[tl]{0.45\textwidth}
        \centering
        \includegraphics[width=\linewidth]{IMG_20220917_204409.jpg}
    \end{subfigure}
    \begin{subfigure}[tr]{0.45\textwidth}
        \centering
        \includegraphics[width=\linewidth]{IMG_20220917_204428.jpg}
    \end{subfigure}
    \begin{subfigure}[bl]{0.45\textwidth}
        \centering
        \includegraphics[width=\linewidth]{IMG_20220917_204619.jpg}
    \end{subfigure}
    \begin{subfigure}[br]{0.45\textwidth}
        \centering
        \includegraphics[width=\linewidth]{IMG_20220917_204801.jpg}
    \end{subfigure}
    \caption{\sffamily Circuito en protoboard}
    \label{fig:proto}
\end{figure}

\section{Conclusión}
{\sffamily\large
    \hspace{0.5cm} Realizar el proyecto 3 fue complicado debido a todas las compuertas lógicas y conexiones que se tenian que hacer, pero esta vez, la \emph{GAL22v10} facilitó mucho las cosas, pues todos los circuitos lógicos se sustituyeron por esta, por lo que se necesito menos espacio y fue más rápido conectar todo.
}

\end{document}
